\documentclass{article}
\usepackage[usenames,dvipsnames]{xcolor}
\usepackage{array}
\usepackage{eurosym}
\usepackage{minted}
\usepackage{float}
\usepackage{amsmath}
\usepackage{color}
\usepackage{graphicx}
\usepackage[utf8]{inputenc}
\usepackage[T1]{fontenc}
\usepackage[english]{babel}
\usepackage[makeroom]{cancel}

\definecolor{bg}{rgb}{0.95,0.95,0.95}

\begin{document}
\title{\textbf{''Introduction to Artificial Intelligence: Homework \#6''}}
\author{LAINE Bastien \#20156441}
\date{Dec. 2th 2015}
\maketitle
\tableofcontents

\newpage
    \section{Problem 1}
        \subsection{A}
            With such a grammar, and for this given sentence, we can find 2 possible parse trees.\\
            The difference between these two is if we consider ``playing'' either as a \textbf{JJ} or as a \textbf{VBZ}.
            \begin{figure}[H]
                \centering
                \includegraphics[scale=0.5]{problem1/graph1.png}
                \caption{$1^{st} graph$}
                \includegraphics[scale=0.5]{problem1/graph2.png}
                \caption{$2^{nd} graph$}
            \end{figure}
            \begin{figure}[H]
                \centering
                \includegraphics[scale=0.5]{problem1/graph2.png}
                \caption{$2^{nd} graph$}
            \end{figure}
        \subsection{B}
            Given the sentence and the grammar we can build a CYK parser table.
            \[
                \begin{tabular}{r|ccccccccc}
                    l=5 & S\\
                    l=4 &  &\\
                    l=3 &  &  & VP\\
                    l=2 & NP &  &  & NP, VP\\
                    l=1 & DT & NN & VBZ & JJ, VBG & NNS \\
                    \hline
                    & an & actress & likes & playing & kids \\
                \end{tabular}
            \]
    \newpage
    \section{Problem 2}
        \subsection{A}
        \subsection{B}
        \subsection{C}
        \subsection{D}
    \newpage
    \section{Problem 3}
        \subsection{A}
            Lets build 2 regular expression, one for URLs and the other for numbers:
            \begin{description}
                \item[URLs]: $(\backslash w+://|[Ww]\{3\}\backslash.)([A-Za-z0-9\_-]+\backslash .)*\backslash .[A-Za-z]\{2,5\}(/[A-Za-z0-9\_-]+)*/?$\\
                    To describe it more in detail:
                    \begin{itemize}
                        \item $(\backslash w+://|[Ww]\{3\}\backslash.)$ represents a pattern beginning by either ``XXX://'', or ``www.''. THis regexp won't recognize URLs like ``google.fr'', too close to some errors like ``Hello.How are you'' (Where Hello.How would be a website).
                        \item $([A-Za-z0-9\_-]+\backslash .)*\backslash .[A-Za-z]\{2,5\}$ represents the main part of the URL. It consists of a 1st part containing $[A-Za-z0-9\_-$], followed by 0+ identical block, separed by a dot. Then it is ended by a sequence of letter (between 2 and 5) (=the extension).
                        \item $(/[A-Za-z0-9\_-]+)*/?$ represents finaly the last part, the remaining part of the adresse, separated by slashes.
                    \end{itemize}
                \item[Phone Numbers]: $\backslash+?[0-9][0-9 -]\{3,\}[0-9]\backslash+?$\\
                    This one is really simple. We just add the cases with ``+'' at the beginning and the end of the number, and we check that the number begins with a digit, and ends with it.\\
                    In the middle, there can be a least 3 digits (+2 at both ends, which make the 5 digits for the smallest phone numbers).\\
                    We also add the possibility to have space or syphen inside the number.
            \end{description}
            On top of that, by looking at the training test case, we can see that there are a lot of price in spam cases. We will add a 3rd regexp for it.\\
            (Important: since I'm on Linux, I have some problem with the file encoding. For this exercise, I've replace every ``£'' by a ``\$'' with another software -Vim-)
            \begin{description}
                \item[Price]: $[$\euro£\$$][0-9]+([,\backslash.][0-9]+)?$
            \end{description}

        \subsection{B}
        \subsection{C}
        \subsection{D}
        \subsection{E}
        \subsection{F}
        \subsection{G}
\end{document}
