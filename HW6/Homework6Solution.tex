\documentclass{article}
\usepackage[usenames,dvipsnames]{xcolor}
\usepackage{array}
\usepackage{minted}
\usepackage{float}
\usepackage{amsmath}
\usepackage{color}
\usepackage{graphicx}
\usepackage[utf8]{inputenc}
\usepackage[T1]{fontenc}
\usepackage[english]{babel}
\usepackage[makeroom]{cancel}

\definecolor{bg}{rgb}{0.95,0.95,0.95}

\begin{document}
\title{\textbf{''Introduction to Artificial Intelligence: Homework \#6''}}
\author{LAINE Bastien \#20156441}
\date{Dec. 2th 2015}
\maketitle
\tableofcontents

\newpage
    \section{Problem 1}
        \subsection{A}
            With such a grammar, and for this given sentence, we can find 2 possible parse trees.\\
            The difference between these two is if we consider ``playing'' either as a \textbf{JJ} or as a \textbf{VBZ}.
            \begin{figure}[H]
                \centering
                \includegraphics[scale=0.5]{problem1/graph1.png}
                \caption{$1^{st} graph$}
                \includegraphics[scale=0.5]{problem1/graph2.png}
                \caption{$2^{nd} graph$}
            \end{figure}
            \begin{figure}[H]
                \centering
                \includegraphics[scale=0.5]{problem1/graph2.png}
                \caption{$2^{nd} graph$}
            \end{figure}
        \subsection{B}
            Given the sentence and the grammar we can build a CYK parser table.
            \[
                \begin{tabular}{r|ccccccccc}
                    l=5 & S\\
                    l=4 &  &\\
                    l=3 &  &  & VP\\
                    l=2 & NP &  &  & NP, VP\\
                    l=1 & DT & NN & VBZ & JJ, VBG & NNS \\
                    \hline
                    & an & actress & likes & playing & kids \\
                \end{tabular}
            \]
    \section{Problem 2}
        \subsection{A}
        \subsection{B}
        \subsection{C}
        \subsection{D}
    \section{Problem 3}
        \subsection{A}
        \subsection{B}
        \subsection{C}
        \subsection{D}
        \subsection{E}
        \subsection{F}
        \subsection{G}
\end{document}
