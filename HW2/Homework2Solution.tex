\documentclass{article}
\usepackage[usenames,dvipsnames]{xcolor}
\usepackage{array}
\usepackage{amsmath}
\usepackage{color}
\usepackage{minted}
\usepackage[utf8]{inputenc}
\usepackage[T1]{fontenc}
\usepackage[english]{babel}
\usepackage[makeroom]{cancel}

\definecolor{bg}{rgb}{0.95,0.95,0.95}

\begin{document}
    \title{\textbf{''Introduction to Artificial Intelligence: Homework \#2''}}
    \author{LAINE Bastien \#20156441}
    \date{Oct. 8th 2015}
    \maketitle
    \tableofcontents

    \newpage

    \section{Problem 1}
        \subsection{1-A}
            To begin with, let's set some variable:
            \begin{itemize}
                \item F stands for the place of Frank
                \item C stands for the place of Clark
                \item K stands for the place of Karen
                \item I stands for the place of Isabel
                \item E stands for the place of Elly
            \end{itemize}
            Having so, we can then represent every contraints by the following token
            \begin{enumerate}
                \item $F\ne C\ne K\ne I\ne E$
                \item $|C-K|\ne 1$
                \item $|E-K|\ne 1$\\
                    $|E-F|\ne 1$
                \item $|I-K|\ne 1$\\
                    $|I-E|\ne 1$
                \item $I-F=1$
                \item $C-I=1$
            \end{enumerate}
        \subsection{1-B}
            After enforcing arc consistency, the remaining value of the domain are:\\
            $\begin{array}{|c|c|c|c|c|}
                \hline
                F&C&K&I&E\\
                \hline
                1,2,3,\xcancel{4},\xcancel{5}&\xcancel{1},\xcancel{2},3,4,5&1,2,\xcancel{3},\xcancel{4},5&\xcancel{1},2,3,4,\xcancel{5}&1,\xcancel{2},\xcancel{3},4,5\\
                \hline
            \end{array}$
        \subsection{1-C}
            But let's choose ``More-constraining-constraint'' which lead us to choose variable $I$, which have 4 constraint\\
            Then we will choose the `Less-constraint-value`` : 3
        \subsection{1-D}

            After selection the value 3 for the variable $I$, our domain looks like that:\\
            $\begin{array}{|c|c|c|c|c|}
                \hline
                F&C&K&I&E\\
                \hline
                1,2,3,\xcancel{4},\xcancel{5}&\xcancel{1},\xcancel{2},3,4,5&1,2,\xcancel{3},\xcancel{4},5&\xcancel{1},\xcancel{2},3,\xcancel{4},\xcancel{5}&1,\xcancel{2},\xcancel{3},4,5\\
                \hline
            \end{array}$\\

            Then according to contraints \#5 and \#6, we can select $F$ and $C$\\
            $\begin{array}{|c|c|c|c|c|}
                \hline
                F&C&K&I&E\\
                \hline
                \xcancel{1},2,\xcancel{3},\xcancel{4},\xcancel{5} & \xcancel{1},\xcancel{2},\xcancel{3},4,\xcancel{5} & 1,2,\xcancel{3},\xcancel{4},5 & \xcancel{1},\xcancel{2},3,\xcancel{4},\xcancel{5} & 1,\xcancel{2},\xcancel{3},4,5\\
                \hline
            \end{array}$\\

            Then by applying constraint \#1 and \#2, we can establish the final domain of our problem:\\
            $\begin{array}{|c|c|c|c|c|}
                \hline
                F&C&K&I&E\\
                \hline
                \xcancel{1},2,\xcancel{3},\xcancel{4},\xcancel{5} & \xcancel{1},\xcancel{2},\xcancel{3},4,\xcancel{5} & 1,\xcancel{2},\xcancel{3},\xcancel{4},\xcancel{5} & \xcancel{1},\xcancel{2},3,\xcancel{4},\xcancel{5} & \xcancel{1},\xcancel{2},\xcancel{3},\xcancel{4},5\\
                \Downarrow&\Downarrow&\Downarrow&\Downarrow&\Downarrow\\
                2&4&1&3&5\\
                \hline
            \end{array}$
    \newpage
    \section{Problem 2}
        \subsection{2-A}
            \subsubsection{Code}
                \inputminted[linenos=true]{python}{square.py}
            \subsubsection{Results without ordering constraint}
                \begin{quote}
                    Length of squares\\
                    (1, 2, 3, 4, 5, 8, 9, 14, 16, 18, 20, 29, 30, 31, 33, 35, 38, 39, 43, 51, 55, 56, 64, 81, 175)\\
                    The length of the 17th square is 38\\
                    Time taken: 0.4
                \end{quote}
            \subsubsection{Results with ordering constraint}
                \begin{quote}
                    Length of squares\\
                    (1, 2, 3, 4, 5, 8, 9, 14, 16, 18, 20, 29, 30, 31, 33, 35, 38, 39, 43, 51, 55, 56, 64, 81, 175)\\
                    The length of the 17th square is 38\\
                    Time taken: 0.0
                \end{quote}


        \subsection{2-B}
            The addition of the ordering constraint make the search faster (from 0.4s to 0.0s).\\
            Indeed, by adding this contraint, we delete a lot of branch during the search.

            For instance, if we try to look at $squares[2]=k$, that mean we can avoid to test every $squares[2+i]<k+i$

            To represent by an array:\\
            $\begin{array}{|c|c|c|c|c|c|c|c|c|c|c|c|c|c|c|}
                \hline
                square\backslash size&1&2&\dots&k-2&k-1&k&k+1&k+2&k+3&k+4&\dots&top-1&top\\
                \hline
                squares[2]&\textcolor{red}{N}&\textcolor{red}{N}&\dots&\textcolor{red}{N}&\textcolor{red}{N}&\textcolor{green}{Y}&\textcolor{red}{N}&\textcolor{red}{N}&\textcolor{red}{N}&\textcolor{red}{N}&\dots&\textcolor{red}{N}&\textcolor{red}{N}\\
                squares[3]&\textcolor{red}{N}&\textcolor{red}{N}&\dots&\textcolor{red}{N}&\textcolor{red}{N}&\textcolor{red}{N}&\textcolor{orange}{C}&\textcolor{orange}{C}&\textcolor{orange}{C}&\textcolor{orange}{C}&\dots&\textcolor{orange}{C}&\textcolor{orange}{C}\\
                squares[4]&\textcolor{red}{N}&\textcolor{red}{N}&\dots&\textcolor{red}{N}&\textcolor{red}{N}&\textcolor{red}{N}&\textcolor{red}{N}&\textcolor{orange}{C}&\textcolor{orange}{C}&\textcolor{orange}{C}&\dots&\textcolor{orange}{C}&\textcolor{orange}{C}\\
                squares[5]&\textcolor{red}{N}&\textcolor{red}{N}&\dots&\textcolor{red}{N}&\textcolor{red}{N}&\textcolor{red}{N}&\textcolor{red}{N}&\textcolor{red}{N}&\textcolor{orange}{C}&\textcolor{orange}{C}&\dots&\textcolor{orange}{C}&\textcolor{orange}{C}\\
                squares[6]&\textcolor{red}{N}&\textcolor{red}{N}&\dots&\textcolor{red}{N}&\textcolor{red}{N}&\textcolor{red}{N}&\textcolor{red}{N}&\textcolor{red}{N}&\textcolor{red}{N}&\textcolor{orange}{C}&\dots&\textcolor{orange}{C}&\textcolor{orange}{C}\\
                squares[7]&\textcolor{red}{N}&\textcolor{red}{N}&\dots&\textcolor{red}{N}&\textcolor{red}{N}&\textcolor{red}{N}&\textcolor{red}{N}&\textcolor{red}{N}&\textcolor{red}{N}&\textcolor{red}{N}&\dots&\textcolor{orange}{C}&\textcolor{orange}{C}\\
                \vdots&\vdots&\vdots&\vdots&\vdots&\vdots&\vdots&\vdots&\vdots&\vdots&\vdots&\vdots&\vdots&\vdots\\
                \hline
            \end{array}$

            Where:
            \begin{itemize}
                \item \textcolor{red}{N} means ``no need to be tested'' (i.e. we are sure it won't work)
                \item \textcolor{green}{Y} means ``no need to be tested'' (i.e. our assumption)
                \item \textcolor{orange}{C} means ``to be checked''
            \end{itemize}

            We can see that with almost no calculus, we can eliminate a huge quantity of test cases.



\end{document}
