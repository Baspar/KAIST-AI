\documentclass{article}
\usepackage[usenames,dvipsnames]{xcolor}
\usepackage{array}
\usepackage{float}
\usepackage{amsmath}
\usepackage{color}
\usepackage{graphicx}
\usepackage[utf8]{inputenc}
\usepackage[T1]{fontenc}
\usepackage[english]{babel}
\usepackage[makeroom]{cancel}

\definecolor{bg}{rgb}{0.95,0.95,0.95}

\begin{document}
\title{\textbf{''Introduction to Artificial Intelligence: Homework \#4''}}
\author{LAINE Bastien \#20156441}
\date{Nov. 6th 2015}
\maketitle
\tableofcontents

\newpage
    \section{Problem 1}
        \subsection{a.}
            The initial state description is:\\
            $At(Monkey,[A,Low],S_0)\land At(Banana,[B,High],S_0)\land At(Box,[C,Low],S_0)$
        \subsection{b.}
            \begin{itemize}
                \item \textbf{Go}
                    \begin{itemize}
                        \item \textbf{Possibility Axiom}: $At(Monkey, p_1, s)\land Equals(Height(p_1), Low)\land Equals(Height(p_2), Low)\Rightarrow Poss(Go(p_1, p_2), s)$
                        \item \textbf{Effect Axiom}: $Poss(Go(p_1, p_2), s)\Rightarrow At(Monkey, p_2, Result(Go(p_1, p_2), s))$
                    \end{itemize}
                \item \textbf{Push}
                    \begin{itemize}
                        \item \textbf{Possibility Axiom}: $Object(obj)\land Equals(Height(p), Low)\land At(Monkey,p_{monkey}, s)\land At(obj, p_{obj}, s)\land Equals(p_{monkey}, p_{obj})\land Equals(Height(p_{monkey}), Low) \Rightarrow Poss(Push(obj, p), s)$
                        \item \textbf{Effect Axiom}: $Poss(Push(obj, p), s)\Rightarrow At(Monkey, p, Result(Push(obj, p), s)) \land At(obj, p, Result(Push(obj, p), s))$
                    \end{itemize}
                \item \textbf{ClimbUp}
                    \begin{itemize}
                        \item \textbf{Possibility Axiom}: $Box(box)\land At(Monkey, p_{monkey}, s)\land At(Box, p_{box}, s)\land Equals(p_{monkey}, p_{box})\land Equals(Height(p_{monkey}), Low)\Rightarrow Poss(ClimbUp(box), s)$
                        \item \textbf{Effect Axiom}: $Poss(ClimbUp(box), s)\Rightarrow$\\
                            $At(Monkey, [Location(p_{box}), High], Result(ClimbUp(box), s))$
                    \end{itemize}
                \item \textbf{ClimbDown}
                    \begin{itemize}
                        \item \textbf{Possibility Axiom}: $Box(box)\land At(Monkey, p_{monkey}, s)\land At(Box, p_{box}, s)\land Equals(p_{monkey}, p_{box})\land Equals(Height(p_{monkey}), High)\Rightarrow Poss(ClimbDown(box), s)$
                        \item \textbf{Effect Axiom}: $Poss(ClimbDown(box), s)\Rightarrow$\\
                            $At(Monkey, p_{box}, Result(ClimbDown(box), s))$
                    \end{itemize}
                \item \textbf{Grasp}
                    \begin{itemize}
                        \item \textbf{Possibility Axiom}: $Banana(banana)\land At(Monkey, p, s)\land At(banana, p, s)\Rightarrow Poss(Grasp(banana), s)$
                        \item \textbf{Effect Axiom}: $Poss(Grasp(banana), s)\Rightarrow Holding(banana, Result(Grasp(banana), s))$
                    \end{itemize}
                \item \textbf{Ungrasp}
                    \begin{itemize}
                        \item \textbf{Possibility Axiom}: $Holding(banana, s)\Rightarrow Poss(Ungrasp(banana), s)$
                        \item \textbf{Effect Axiom}: $Poss(Ungrasp(banana), s)\Rightarrow \lnot Holding(banana, Result(Ungrasp(banana), s))$
                    \end{itemize}
            \end{itemize}
        \subsection{c.}
            The general goal is:\\
            $At(Box, p_{init}, s_0)\land At(Box, p_{init}, s)\land Holding(Banana, s)$
            %TODO: last part of the question
        \subsection{d.}
            Since we want to prohip moving an object if it's too heavy, we'll just change the \textbf{Possibility Axiom} by adding $\lnot Heavy(object)$, leading to:
            \begin{itemize}
                \item \textbf{Push}
                    \begin{itemize}
                        \item \textbf{Possibility Axiom}: $Object(obj)\land \lnot Heavy(obj)\land Equals(Height(p), Low)\land At(Monkey,p_{monkey}, s)\land At(obj, p_{obj}, s)\land Equals(p_{monkey}, p_{obj})\land Equals(Height(p_{monkey}), Low) \Rightarrow Poss(Push(obj, p), s)$
                        \item \textbf{Effect Axiom}: $Poss(Push(obj, p), s)\Rightarrow At(Monkey, p, Result(Push(obj, p), s)) \land At(obj, p, Result(Push(obj, p), s))$
                    \end{itemize}
            \end{itemize}
    \section{Problem 2}
    \section{Problem 3}
        \subsection{a}
            For this part, we just have to copy/paste the entire code of the page.\\
            Since the used version in the tutorial is older than mine, I just added ``:-dynamic(on/2)'' to enable the retract of on() clauses.\\
            The final code is included in the file p3.a.pl\\
            The different log are:\\
            \begin{tabular}{|p{6cm}|p{6cm}|}
                \hline
                    \textbf{Commands} & \textbf{Log}\\
                \hline
                    put\_on() & ?- put\_on(a,table).
                    \newline \textcolor{gray}{    true .}
                    \newline ?- listing(on), listing(move).
                    \newline \textcolor{gray}{:- dynamic on/2.}
                    \newline \textcolor{gray}{    on(b, c).}
                    \newline \textcolor{gray}{    on(c, table).}
                    \newline \textcolor{gray}{    on(a, table).}
                    \newline \textcolor{gray}{:- dynamic move/3.}
                    \newline \textcolor{gray}{    move(a, b, table).}
                    \newline \textcolor{gray}{    true.}
                    \newline ?- put\_on(c,a).
                    \newline \textcolor{gray}{    false.}\\
                \hline
                    r\_put\_on() & ?- r\_put\_on(c,a).
                    \newline \textcolor{gray}{    true .}
                    \newline ?- listing(on), listing(move).
                    \newline \textcolor{gray}{:- dynamic on/2.}
                    \newline \textcolor{gray}{    on(a, table).}
                    \newline \textcolor{gray}{    on(b, table).}
                    \newline \textcolor{gray}{    on(c, a).}
                    \newline \textcolor{gray}{:- dynamic move/3.}
                    \newline \textcolor{gray}{    move(a, b, table).}
                    \newline \textcolor{gray}{    move(b, c, table).}
                    \newline \textcolor{gray}{    move(c, table, a).}
                    \newline \textcolor{gray}{    true.}\\
                \hline
                    do() & ?- do([on(a,table),on(b,a),on(c,b)]).
                    \newline \textcolor{gray}{    true .}
                    \newline ?- listing(on), listing(move).
                    \newline \textcolor{gray}{:- dynamic on/2.}
                    \newline \textcolor{gray}{    on(a, table).}
                    \newline \textcolor{gray}{    on(b, a).}
                    \newline \textcolor{gray}{    on(c, b).}
                    \newline \textcolor{gray}{:- dynamic move/3.}
                    \newline \textcolor{gray}{    move(a, b, table).}
                    \newline \textcolor{gray}{    move(b, c, a).}
                    \newline \textcolor{gray}{    move(c, table, b).}
                    \newline \textcolor{gray}{    true.}\\
                \hline
            \end{tabular}
        \subsection{b}
            For this part, both codes can be found on files p3.b1.pl and p3.b2.pl.\\
            The two last command are commented, they are the one to execute by hand on the command line interface.\\
            \begin{tabular}{|p{3cm}|p{6cm}|p{3cm}|}
                \hline
                \textbf{Initialization} & \textbf{Goal} & \textbf{Log}\\
                \hline
                    on(a,b).\newline on(b,c).\newline on(c,d).\newline on(d,e).\newline on(e,table). &
                    [on(a, table), on(b, table), on(c, table), on(d, table), on(e, table)].&
                    :- dynamic on/2.\newline on(e, table).\newline on(a, table).\newline on(b, table).\newline on(c, table).\newline on(d, table).\newline :- dynamic move/3.\newline move(a, b, table).\newline move(b, c, table).\newline move(c, d, table).\newline move(d, e, table).\newline true.\\
                \hline%TODO: Change 2nd exeample
                    on(a,b).\newline on(b,c).\newline on(c,d).\newline on(d,e).\newline on(e,table). &
                    [on(a, table), on(b, table), on(c, table), on(d, table), on(e, table)].&
                    :- dynamic on/2.\newline on(e, table).\newline on(a, table).\newline on(b, table).\newline on(c, table).\newline on(d, table).\newline :- dynamic move/3.\newline move(a, b, table).\newline move(b, c, table).\newline move(c, d, table).\newline move(d, e, table).\newline true.\\
                \hline
            \end{tabular}

    \section{Problem 4}
\end{document}
