\documentclass{article}
\usepackage{array}
\usepackage{amsmath}
\usepackage{minted}
\usepackage[utf8]{inputenc}
\usepackage[T1]{fontenc}
\usepackage[english]{babel}
\usepackage[makeroom]{cancel}

\newcolumntype{L}{>{\centering\arraybackslash}m{2cm}}


\begin{document}
    \title{\textbf{''Introduction to Artificial Intelligence: Homework \#2''}}
    \author{LAINE Bastien \#20156441}
    \date{Sept. 16th 2015}
    \maketitle
    \tableofcontents

    \newpage

    \section{Problem 1}
        \subsection{1-A}
            To begin with, let's set some variable:
            \begin{itemize}
                \item F stands for the place of Frank
                \item C stands for the place of Clark
                \item K stands for the place of Karen
                \item I stands for the place of Isabel
                \item E stands for the place of Elly
            \end{itemize}
            Having so, we can then represent every contraints by the following token
            \begin{enumerate}
                \item $F\ne C\ne K\ne I\ne E$
                \item $|C-K|\ne 1$
                \item $|E-K|\ne 1$\\
                    $|E-F|\ne 1$
                \item $|I-K|\ne 1$\\
                    $|I-E|\ne 1$
                \item $I-F=1$
                \item $C-I=1$
            \end{enumerate}
        \subsection{1-B}
            After enforcing arc consistency, the remaining value of the domain are:\\
            $\begin{array}{|c|c|c|c|c|}
                \hline
                F&C&K&I&E\\
                \hline
                1,2,3,\xcancel{4},\xcancel{5}&\xcancel{1},\xcancel{2},3,4,5&1,2,\xcancel{3},\xcancel{4},5&\xcancel{1},2,3,4,\xcancel{5}&1,\xcancel{2},\xcancel{3},4,5\\
                \hline
            \end{array}$
        \subsection{1-C}
            For this example, the choice of heuristic doesn't really matters, since as soon as we choose a variable, and a value for it, the system will be totally constraint.\\
            But let's choose ``Least-constraining-value'' which lead us to choose variable $F$
        \subsection{1-D}
    \section{Problem 2}
        \subsection{2-A}
            %See code in file $squares.py$.
            \inputminted{python}{square.py}
        \subsection{2-B}
            The addition of the orderig constraint make the search faster (from 0.4s to 0.0s).\\
            Indeed, by adding this contraint, we delete a lot of branch during the search. For instance

\end{document}
